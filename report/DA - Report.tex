%%%%%%%%%%%%%%%%%%%%%%%%%%%%%%%%%%%%%%%%%
% Arsclassica Article
% LaTeX Template
% Version 1.1 (10/6/14)
%
% This template has been downloaded from:
% http://www.LaTeXTemplates.com
%
% Original author:
% Lorenzo Pantieri (http://www.lorenzopantieri.net) with extensive modifications by:
% Vel (vel@latextemplates.com)
%
% License:
% CC BY-NC-SA 3.0 (http://creativecommons.org/licenses/by-nc-sa/3.0/)
%
%%%%%%%%%%%%%%%%%%%%%%%%%%%%%%%%%%%%%%%%%

%----------------------------------------------------------------------------------------
%	PACKAGES AND OTHER DOCUMENT CONFIGURATIONS
%----------------------------------------------------------------------------------------

\documentclass[
10pt, % Main document font size
a4paper, % Paper type, use 'letterpaper' for US Letter paper
oneside, % One page layout (no page indentation)
%twoside, % Two page layout (page indentation for binding and different headers)
headinclude,footinclude, % Extra spacing for the header and footer
BCOR5mm, % Binding correction
]{scrartcl}

%%%%%%%%%%%%%%%%%%%%%%%%%%%%%%%%%%%%%%%%%
% Arsclassica Article
% Structure Specification File
%
% This file has been downloaded from:
% http://www.LaTeXTemplates.com
%
% Original author:
% Lorenzo Pantieri (http://www.lorenzopantieri.net) with extensive modifications by:
% Vel (vel@latextemplates.com)
%
% License:
% CC BY-NC-SA 3.0 (http://creativecommons.org/licenses/by-nc-sa/3.0/)
%
%%%%%%%%%%%%%%%%%%%%%%%%%%%%%%%%%%%%%%%%%

%----------------------------------------------------------------------------------------
%	REQUIRED PACKAGES
%----------------------------------------------------------------------------------------

\usepackage[
nochapters, % Turn off chapters since this is an article        
beramono, % Use the Bera Mono font for monospaced text (\texttt)
eulermath,% Use the Euler font for mathematics
pdfspacing, % Makes use of pdftex’ letter spacing capabilities via the microtype package
dottedtoc % Dotted lines leading to the page numbers in the table of contents
]{classicthesis} % The layout is based on the Classic Thesis style

\usepackage{arsclassica} % Modifies the Classic Thesis package

\usepackage[T1]{fontenc} % Use 8-bit encoding that has 256 glyphs

\usepackage[utf8]{inputenc} % Required for including letters with accents

\usepackage{graphicx} % Required for including images
\graphicspath{{Figures/}} % Set the default folder for images

\usepackage{enumitem} % Required for manipulating the whitespace between and within lists

\usepackage{lipsum} % Used for inserting dummy 'Lorem ipsum' text into the template

\usepackage{subfig} % Required for creating figures with multiple parts (subfigures)

\usepackage{amsmath,amssymb,amsthm} % For including math equations, theorems, symbols, etc

\usepackage{varioref} % More descriptive referencing

%----------------------------------------------------------------------------------------
%	THEOREM STYLES
%---------------------------------------------------------------------------------------

\theoremstyle{definition} % Define theorem styles here based on the definition style (used for definitions and examples)
\newtheorem{definition}{Definition}

\theoremstyle{plain} % Define theorem styles here based on the plain style (used for theorems, lemmas, propositions)
\newtheorem{theorem}{Theorem}

\theoremstyle{remark} % Define theorem styles here based on the remark style (used for remarks and notes)

%----------------------------------------------------------------------------------------
%	HYPERLINKS
%---------------------------------------------------------------------------------------

\hypersetup{
%draft, % Uncomment to remove all links (useful for printing in black and white)
colorlinks=true, breaklinks=true, bookmarks=true,bookmarksnumbered,
urlcolor=webbrown, linkcolor=RoyalBlue, citecolor=webgreen, % Link colors
pdftitle={}, % PDF title
pdfauthor={\textcopyright}, % PDF Author
pdfsubject={}, % PDF Subject
pdfkeywords={}, % PDF Keywords
pdfcreator={pdfLaTeX}, % PDF Creator
pdfproducer={LaTeX with hyperref and ClassicThesis} % PDF producer
} % Include the structure.tex file which specified the document structure and layout

\hyphenation{Fortran hy-phen-ation} % Specify custom hyphenation points in words with dashes where you would like hyphenation to occur, or alternatively, don't put any dashes in a word to stop hyphenation altogether
\usepackage{pgfgantt}
\usepackage{lipsum}
\usepackage{float}
\usepackage{caption}
\usepackage{ragged2e}


%----------------------------------------------------------------------------------------
%	TITLE AND AUTHOR(S)
%----------------------------------------------------------------------------------------

\title{\normalfont\spacedallcaps{Data Analytics Coursework: Predict the 2017 NCAA Basketball Tournament}} % The article title
\author{\spacedlowsmallcaps{Ivaylo Snezhanov Shalev, Ziqi Zhao, Yuan-Yi Chang}} % The article author(s) - author affiliations need to be specified in the AUTHOR AFFILIATIONS block

\date{} % An optional date to appear under the author(s)

%----------------------------------------------------------------------------------------

\begin{document}

%----------------------------------------------------------------------------------------
%	HEADERS
%----------------------------------------------------------------------------------------

\renewcommand{\sectionmark}[1]{\markright{\spacedlowsmallcaps{#1}}} % The header for all pages (oneside) or for even pages (twoside)
%\renewcommand{\subsectionmark}[1]{\markright{\thesubsection~#1}} % Uncomment when using the twoside option - this modifies the header on odd pages
\lehead{\mbox{\llap{\small\thepage\kern1em\color{halfgray} \vline}\color{halfgray}\hspace{0.5em}\rightmark\hfil}} % The header style

\pagestyle{scrheadings} % Enable the headers specified in this block

%----------------------------------------------------------------------------------------
%	TABLE OF CONTENTS & LISTS OF FIGURES AND TABLES
%----------------------------------------------------------------------------------------

\maketitle % Print the title/author/date block

%\setcounter{tocdepth}{2} % Set the depth of the table of contents to show sections and subsections only

%\tableofcontents % Print the table of contents

%\listoffigures % Print the list of figures

%\listoftables % Print the list of tables

%----------------------------------------------------------------------------------------
%	ABSTRACT
%----------------------------------------------------------------------------------------

%\section*{Abstract} % This section will not appear in the table of contents due to the star (\section*)

%\lipsum[1] % Dummy text

%----------------------------------------------------------------------------------------
%	AUTHOR AFFILIATIONS
%----------------------------------------------------------------------------------------

%{\let\thefootnote\relax\footnotetext{* \textit{Department of Biology, University of Examples, London, United Kingdom}}}

%{\let\thefootnote\relax\footnotetext{\textsuperscript{1} \textit{Department of Chemistry, University of Examples, London, United Kingdom}}}

%----------------------------------------------------------------------------------------

%\newpage % Start the article content on the second page, remove this if you have a longer abstract that goes onto the second page

%----------------------------------------------------------------------------------------
%	Problem description
%----------------------------------------------------------------------------------------

\section{Problem description}
 
%----------------------------------------------------------------------------------------
%	Dataset
%----------------------------------------------------------------------------------------

\section{Dataset}
After analysing data sets from different sources we decided to use the KenPom statistical data of all US basketball teams for the past 15 years (from 2002). As helping dataset of the teams playing this year we used the kaggle's dataset.
\subsection{Features description}
The KenPom statistical data includes the years of 2002 until 2016 and includes 10 features. This data is collected and calculated based on the idea of the four factor concept from Dean Oliver.
Oliver believes four factors determine why a team wins or loses. The four factors are:
\begin{enumerate}
\item Shooting
\item Avoiding turnovers
\item Offensive rebounding
\item Getting to the foul line
\end{enumerate}
A team is measured both offensively and defensively on these four factors.
\subsubsection{Number of wins and losses (W-L)}
This is aggregative value of the total games won (W) and lost (L) for the team, for the specific season (year). As we are trying to predict if team A will win against team B for the upcomming tournament (season) we can't use this feature as we will not know the wins and loses for the tested new season.
\subsubsection{Adjusted Efficiency Margin (AdjEM)}
This is the difference between a team’s offensive and defensive efficiency.
\[AdjEM = AdjOE - AdjDE\]
AdjEM is used to rate teams. It was introduced by KenPom website prior to the 2016-2017 season. Prior to this metric, ratings were built on log5 and the Pythagorean expectation.

The adjusted efficiency margin shows the number of points a team would be expected to outscore the average Division-I team over 100 possessions without adjusting for location of the game.
For example, in 2015-2016, if Villanova played the average Division-I team on a neutral court, on average, Villanova would win a 100-possession game by about 32 points.
Villanova's AdjOE is 121.7 and AdjDE is 89.7.\[AdjEM = 121.7 - 89.7 = 32\] AdjEM is used because it's simpler and it's a linear measure.
The Pythagorean expectation, from the mind of Bill James, originated from baseball. It estimates how many games a team should have won based on the amount of runs it scored or allowed.
It applied to college basketball by giving you a team's expected winning percentage against the average Division-I team.\[Pyth = (AdjOE^{11.5}) / (AdjOE^{11.5} + AdjDE^{11.5})\]
It's clear the Pythagorean expectation is complicated. AdjEM is subtraction. That's it. A team's offensive efficiency minus its defensive efficiency.
The difference when comparing teams using AdjEM versus log5 or the Pythagorean expectation is important too. The Pythagorean expectation for one team could be .96 and for another could be .93. The difference of .3 there is not the same as comparing teams with a .70 and .67 expected winning percentage because of the team's strength. It's murky.
KenPom describes AdjEM as a linear measure giving the example of a team with a +31 AdjEM and a team with a +28 AdjEM the same as the difference between +4 and +1. It's more clear.\cite{adjem}

\subsubsection{Adjusted offensive efficiency (AdjO)}
A team's offensive efficiency is the amount of points it scores per 100 offensive possessions.
\[OE = (Points scored * 100) / Possessions\]
A team's pace is determined by how they like to play and how their opponents like to play. This is the reason efficiency numbers need to be adjusted. It accounts for competition or how a team and their opponents want to play.

\subsubsection{Adjusted defensive efficiency (AdjD)}
A team's defensive efficiency is the amount of points it allows per 100 defensive possessions.
\[DE = (Points allowed * 100) / Possessions\]

\subsubsection{Adjusted tempo (AdjT)}
Possessions per 40 minutes. In every game, each team wants to play at a certain pace. Adjusted tempo tries to tell you the pace each team wants to play. It's an estimate of the pace a team would have against the team that wants to play at an average Division-I tempo.

\subsubsection{Luck rating (Luck)}
Luck doesn't factor into a team's rating. It's a metric that compares a team's record to what they deserved based on their game-by-game efficiency.
If a team is involved in a lot of close games, it shouldn't win or lose all of them. If a team wins all of the close games, they're viewed as a lucky. An unlucky team would lose all of their close games.
Luck is measured using Dean Oliver's correlated Gaussian method.
If this sounds complicated, it definitely is.
Oliver explains it as something similar to a bell curve.
\begin{equation}
Win\% = \mathit{NORM}\left [ \frac{\left (Rtg-Opp.Rtg \right )}{\mathit{SD}\left ( Rating Difference \right )} \right ]
\end{equation}
\begin{equation}
\begin{split}
\mathit{SD}\left(Rating Difference\right) & = \mathit{SD}\left(Rtg - Opp.Rtg\right)  \\
							& = \mathit{SQRT}\left(\mathit{Var}\left(Rtg\right)+\mathit{Var}\left(Opp.Rtg\right) - 2*\mathit{Cov}\left(Rtg,Opp.Rtg\right)\right)
\end{split}
\end{equation}
The components:
\begin{enumerate}
\item Rtg: points scored per 100 possessions (offensive rating)
\item Opp.Rtg: points allowed per 100 possessions (defensive rating)
\item \textit{SD}: statistical standard deviation of quantity in parentheses ()
\item \textit{Var}: statistical variance of quantity in parentheses ()
\item \textit{Cov}: statistical covariance of quantities in parentheses ()
\end{enumerate}

Luck tells you the difference between expected winning percentage and its actual winning percentage.
Luck doesn't use Pythagorean Winning Percentage (Pyth) as the expected winning percentage. Pyth is calculated by a team's offensive and defensive efficiencies.
The correlated Gaussian method uses the distribution of a team's game efficiencies to determine the expected winning percentage. It includes both the average margin of victory and the variation in a team's margin of victory.
This method takes into account that the majority or teams play to the level of their competition.

\begin{description}
\item[Example]
In 2015-2016, Hampton was viewed as the second luckiest team in Division-I.
Hampton was 21-11. It's actual winning percentage was .656.
Using the correlated Gaussian method, Hampton's expect winning percentage was .499.
Hampton's actual winning percentage is .157 points higher than its expected. This translates into roughly 5 wins that are attributed to luck.
If you take a glance at Hampton's results, it played 3 overtime games and one double-overtime game. It won all 4 of these contests, which can be viewed as lucky.
In comparison, the 2015-2016 Clemson Tigers finished 17-14. Clemson was ranked 339th out of 351 teams in luck.
Clemson's actual winning percentage was .548.
It's expected winning percentage using Oliver's method was .643.
This is -0.095 points lower than expected. This means almost 3 losses were attributed to luck.
Taking a look at Clemson's game-by-game results, you'll find 10 losses by a 7 points or less. This can be seen as unlucky.
Remember luck doesn't factor into ratings
Luck isn't used when rating a team. It gives you an idea on how a team performs in close game and how it plays to its competition.
A very lucky team will likely be rated lower by KenPom.
Where an extremely unlucky team could be rated higher.
Using our example above, Hampton (lucky) was 229th in the final 2015-2016 KenPom ratings.
While Clemson (unlucky) was 48th in the final 2015-2016 KenPom ratings.\cite{luck}
\end{description}

\subsubsection{Strength of schedule (SOS)}
A team's strength of schedule is made up of 3 components:
\begin{description}
\item[AdjEM] The overall strength of schedule of a team.
\item[AdjO] Opponent's average adjusted offensive efficiency.
\item [AdjD] Opponent's average adjusted defensive efficiency.
\end{description}
AdjEM for strength of schedule is calculated:
\[AdjEM of SOS = AdjO - AdjD\]
Prior to the 2016-17 season, KenPom started using AdjEM for ratings and strength of schedule instead of the Pythagorean expectation and log5.
Another change is now KenPom is using Jeff Sagarin's WIN50 Method.

Sagarin's WIN50 method shows the strength of a team that would be expected to win half its games against the team's schedule. It compares all teams on the same scale and reduces the impact of outliers.

The previous method used by KenPom measured a team's strength of schedule by the average of its opponents' rating. This means outliers like playing the 350th or 351st team would have a negative impact on a team's SOS.

The WIN50 method doesn't put too much emphasis on the quality of bad opponents a team plays. It's more fair than previous methods.

Outside of factoring into a team's rating or predicting how it will perform, strength of schedule can be used to rate different conferences as a whole.\cite{sos}

\subsubsection{Adjusted Efficiency Margin (SOS AdjEM)}
The Adjusted Efficiency Margin, but just durring the season.

\subsubsection{Average AdjOE of opposing offenses (SOS OppO)}
The Adjusted Offence Efficiency just durring the season.

\subsubsection{Average AdjDE of opposing defenses (SOS OppD)}
The Adjusted Defence Efficiency just driving the season.

\subsubsection{Non-conference strength of schedule (NCSOS)}
A team does have some control over its schedule. It can choose or agree to its non-conference schedule.
The non-conference strength of schedule uses the 3 components only for this portion of the schedule. It doesn't include post season or conference games.

\begin{enumerate}
\item NCSOS AdjEM is the non-conference strength of schedule of a team. It's calulated using NCSOS AdjO and NCSOS AjdD:
\[NCSOS AdjEM = NCSOS AdjO - NCSOS AjdD\]
\end{enumerate}
%----------------------------------------------------------------------------------------
%	Exploration
%----------------------------------------------------------------------------------------

\section{Exploration}

%----------------------------------------------------------------------------------------
%	Analysis
%----------------------------------------------------------------------------------------

\section{Analysis}

%----------------------------------------------------------------------------------------
%	Result
%----------------------------------------------------------------------------------------

\section{Result}

%----------------------------------------------------------------------------------------
%	Future work
%----------------------------------------------------------------------------------------

\section{Future work}


%----------------------------------------------------------------------------------------
%	BIBLIOGRAPHY
%----------------------------------------------------------------------------------------

\begin{thebibliography}{3}
\bibitem{adjem} 
 Adjusted Efficiency Margin, https://cbbstatshelp.com/ratings/adjem/
 
 \bibitem{luck}
 Luck, https://cbbstatshelp.com/ratings/luck/
 
 \bibitem{sos}
 Overall Strength of Schedule, https://cbbstatshelp.com/ratings/strength-of-schedule
\end{thebibliography}

%----------------------------------------------------------------------------------------



\end{document}